%%%%%%%%%%%%%%%%%%%%%%%%%%%%%%%%%%%%%%%%%%%%%%%%%%%%%%%%%%%%%%%%%%
%%%%%%%%%%%%%%%%%%%%%%%%%%%%%%%%%%%%%%%%%%%%%%%%%%%%%%%%%%%%%%%%%%
% template: https://www.overleaf.com/latex/templates/imperial-assignment-template/pbnrncvckyjv

%Packages
\documentclass[10pt, a4paper]{article}
\usepackage[top=3cm, bottom=4cm, left=3.5cm, right=3.5cm]{geometry}
\usepackage{amsmath,amsthm,amsfonts,amssymb,amscd, fancyhdr, color, comment, graphicx, environ}
\usepackage{float}
\usepackage{mathrsfs}
\usepackage{unicode-math}
% \setmathfont{TeX Gyre Termes Math}
\usepackage{lastpage}
\usepackage[dvipsnames]{xcolor}
\usepackage[framemethod=TikZ]{mdframed}
\usepackage{enumerate}
\usepackage[shortlabels]{enumitem}
\usepackage{fancyhdr}
\usepackage{indentfirst}
\usepackage{listings}
\usepackage{sectsty}
\usepackage{thmtools}
\usepackage{shadethm}
\usepackage{hyperref}
\usepackage{setspace}
\usepackage[linguistics]{forest}
\hypersetup{
    colorlinks=true,
    linkcolor=blue,
    filecolor=magenta,      
    urlcolor=blue,
}
%%%%%%%%%%%%%%%%%%%%%%%%%%%%%%%%%%%%%%%%%%%%%%%%%%%%%%%%%%%%%%%%%%
%%%%%%%%%%%%%%%%%%%%%%%%%%%%%%%%%%%%%%%%%%%%%%%%%%%%%%%%%%%%%%%%%%
%Environment setup
\mdfsetup{skipabove=\topskip,skipbelow=\topskip}
\newrobustcmd\ExampleText{%
An \textit{inhomogeneous linear} differential equation has the form
\begin{align}
L[v ] = f,
\end{align}
where $L$ is a linear differential operator, $v$ is the dependent
variable, and $f$ is a given non−zero function of the independent
variables alone.
}
\mdfdefinestyle{theoremstyle}{%
linecolor=black,linewidth=1pt,%
frametitlerule=true,%
frametitlebackgroundcolor=gray!20,
innertopmargin=\topskip,
}
\mdtheorem[style=theoremstyle]{Problem}{Problem}
\newenvironment{Solution}{\textbf{Solution.}}

\definecolor{codegreen}{rgb}{0,0.6,0}
\definecolor{codegray}{rgb}{0.5,0.5,0.5}
\definecolor{codepurple}{rgb}{0.58,0,0.82}
\definecolor{backcolour}{rgb}{0.95,0.95,0.92}

\lstdefinestyle{mystyle}{
    backgroundcolor=\color{backcolour},   
    commentstyle=\color{codegreen},
    keywordstyle=\color{magenta},
    numberstyle=\tiny\color{codegray},
    stringstyle=\color{codepurple},
    basicstyle=\ttfamily\footnotesize,
    breakatwhitespace=false,         
    breaklines=true,                 
    captionpos=b,                    
    keepspaces=true,                 
    numbers=left,                    
    numbersep=5pt,                  
    showspaces=false,                
    showstringspaces=false,
    showtabs=false,                  
    tabsize=2
}

\lstset{style=mystyle}
%%%%%%%%%%%%%%%%%%%%%%%%%%%%%%%%%%%%%%%%%%%%%%%%%%%%%%%%%%%%%%%%%%
%%%%%%%%%%%%%%%%%%%%%%%%%%%%%%%%%%%%%%%%%%%%%%%%%%%%%%%%%%%%%%%%%%
%Fill in the appropriate information below
\newcommand{\norm}[1]{\left\lVert#1\right\rVert}
\newcommand\course{Machine Learning - Supervised Learning}
\newcommand\hwnumber{M-MLR-900}
\newcommand\Information{(Alexandre Guichet, Alexis Auriac, Benjamin Feller)}
%%%%%%%%%%%%%%%%%%%%%%%%%%%%%%%%%%%%%%%%%%%%%%%%%%%%%%%%%%%%%%%%%%
%%%%%%%%%%%%%%%%%%%%%%%%%%%%%%%%%%%%%%%%%%%%%%%%%%%%%%%%%%%%%%%%%%
%Page setup
\pagestyle{fancy}
\headheight 35pt
\lhead{\today}
\rhead{\includegraphics[width=2.5cm]{epitech_logo.png}}
\lfoot{}
\pagenumbering{arabic}
\cfoot{\small\thepage}
\rfoot{}
\headsep 1.2em
\renewcommand{\baselinestretch}{1.25}
%%%%%%%%%%%%%%%%%%%%%%%%%%%%%%%%%%%%%%%%%%%%%%%%%%%%%%%%%%%%%%%%%%
%%%%%%%%%%%%%%%%%%%%%%%%%%%%%%%%%%%%%%%%%%%%%%%%%%%%%%%%%%%%%%%%%%
%Add new commands here
\renewcommand{\labelenumi}{\alph{enumi})}
\newcommand{\Z}{\mathbb Z}
\newcommand{\R}{\mathbb R}
\newcommand{\Q}{\mathbb Q}
\newcommand{\NN}{\mathbb N}
\newcommand{\PP}{\mathbb P}
\DeclareMathOperator{\Mod}{Mod} 
\renewcommand\lstlistingname{Algorithm}
\renewcommand\lstlistlistingname{Algorithms}
\def\lstlistingautorefname{Alg.}
\newtheorem*{theorem}{Theorem}
\newtheorem*{lemma}{Lemma}
\newtheorem{case}{Case}
\newcommand{\assign}{:=}
\newcommand{\infixiff}{\text{ iff }}
\newcommand{\nobracket}{}
\newcommand{\backassign}{=:}
\newcommand{\tmmathbf}[1]{\ensuremath{\boldsymbol{#1}}}
\newcommand{\tmop}[1]{\ensuremath{\operatorname{#1}}}
\newcommand{\tmtextbf}[1]{\text{{\bfseries{#1}}}}
\newcommand{\tmtextit}[1]{\text{{\itshape{#1}}}}

\newenvironment{itemizedot}{\begin{itemize} \renewcommand{\labelitemi}{$\bullet$}\renewcommand{\labelitemii}{$\bullet$}\renewcommand{\labelitemiii}{$\bullet$}\renewcommand{\labelitemiv}{$\bullet$}}{\end{itemize}}
\catcode`\<=\active \def<{
\fontencoding{T1}\selectfont\symbol{60}\fontencoding{\encodingdefault}}
\catcode`\>=\active \def>{
\fontencoding{T1}\selectfont\symbol{62}\fontencoding{\encodingdefault}}
\catcode`\<=\active \def<{
\fontencoding{T1}\selectfont\symbol{60}\fontencoding{\encodingdefault}}

%%%%%%%%%%%%%%%%%%%%%%%%%%%%%%%%%%%%%%%%%%%%%%%%%%%%%%%%%%%%%%%%%%
%%%%%%%%%%%%%%%%%%%%%%%%%%%%%%%%%%%%%%%%%%%%%%%%%%%%%%%%%%%%%%%%%%
%Begin now!

\begin{document}

\begin{titlepage}
    \begin{center}
        \vspace*{3cm}
            
        \Huge
        \textbf{Report}
            
        \vspace{1cm}
        \huge
        \hwnumber
            
        \vspace{1.5cm}
        \Large
            
        \textbf{\Information}                      % <-- author
        
            
        \vfill
        
        \course
            
        \vspace{1cm}
            
        \includegraphics[width=0.4\textwidth]{epitech_logo.png}

        \Large
        
        \today
            
    \end{center}
\end{titlepage}

%%%%%%%%%%%%%%%%%%%%%%%%%%%%%%%%%%%%%%%%%%%%%%%%%%%%%%%%%%%%%%%%%%
%%%%%%%%%%%%%%%%%%%%%%%%%%%%%%%%%%%%%%%%%%%%%%%%%%%%%%%%%%%%%%%%%%
%Start the assignment now
%%%%%%%%%%%%%%%%%%%%%%%%%%%%%%%%%%%%%%%%%%%%%%%%%%%%%%%%%%%%%%%%%%
\newpage

\subsubsection*{Introduction}

We did part 1, 2, 3, and 4 but did not have time to do part 5.

Division of labor:
\begin{itemize}
    \item Alexandre: part 1
    \item Benjamin: part 2
    \item Alexis: part 3 and 4, writing report
\end{itemize}

\newpage

%%%%%%%%%%%%%%%%%%%%%%%%%%%%%%%%%%%%%%%%%%%%%%%%%%%%%%%%%%%%%%%%%%
%Part 1
%%%%%%%%%%%%%%%%%%%%%%%%%%%%%%%%%%%%%%%%%%%%%%%%%%%%%%%%%%%%%%%%%%

\subsection*{Part 1}

\begin{Problem}
    The goal of this exercise is to work with statistical notions such as mean, standard deviation, and correlation.
    
    Write a file named artificial\_dataset.py that generates a numerical dataset with 300 datapoints (i.e. lines) and at least 6 columns and saves it to a csv file or to a numpy array in a binary python file.

    The columns must satisfy the following requirements:
    \begin{itemize}
        \item they must all have a different mean
        \item they must all have a different standard deviation (English for "écart type")
        \item at least one column should contain integers.
        \item at least one column should contain floats.
        \item one column must have a mean close to 2.5.
        \item some columns must be positively correlated.
        \item some columns must be negatively correlated.
        \item some columns must have a correlation close to 0.
    \end{itemize}
\end{Problem}

\begin{Solution}

See \verb|exercise_1/artificial_dataset.py| for the code.

The generated data can be found in \verb|exercise_1/data.npy|, it contains 300 dataponts with 6 columns.

Let's go over each point mentioned in the subject one by one.

\textbf{All columns must have a different mean}

\begin{itemize}
    \item column 1: 2.58
    \item column 2: 0.898
    \item column 3: 1.686
    \item column 4: 4.707
    \item column 5: 0.349
    \item column 6: 10.115
\end{itemize}

\textbf{All columns must have a different standard deviation}

\begin{itemize}
    \item column 1: 1.752
    \item column 2: 0.827
    \item column 3: 1.998
    \item column 4: 1.889
    \item column 5: 2.430
    \item column 6: 1.352
\end{itemize}

\textbf{At least one column should contain integers}

Column 1 contains integers.

\textbf{At least one column should contain floats}

All columns except column 1 contain floats.

\textbf{One column must have a mean close to 2.5}

Column 1 has a mean of 2.58.

\textbf{Columns correlations}

Using \verb|numpy.corrcoef| to get a correlation matrix for column 4, 5, and 6 we get this:

\hfill

\begin{center}
    $
    \begin{pmatrix}
        1. & -0.5595593 & 0.66913029\\
        -0.5595593 & 1. & 0.03168539\\
        0.66913029 & 0.03168539 & 1.
    \end{pmatrix}
    $
\end{center}

\hfill

Column 4 is \textbf{negatively correlated} with column 5.

Column 4 is \textbf{positively correlated} with column 6.

Column 5 is \textbf{has a correlation close to 0} with column 6.

\end{Solution}

%%%%%%%%%%%%%%%%%%%%%%%%%%%%%%%%%%%%%%%%%%%%%%%%%%%%%%%%%%%%%%%%%%
%Part 2
%%%%%%%%%%%%%%%%%%%%%%%%%%%%%%%%%%%%%%%%%%%%%%%%%%%%%%%%%%%%%%%%%%

\subsection*{Part 2}

\begin{Problem}
    A dataset representing a population is stored in dataset.csv inside the

    \verb|project/ex_2_metric/folder|.

    Define a metric in this dataset, which means define a dissimilarity between the samples, by taking into account all their features (columns of the dataset).

    Some features are numerical and others are categorical, hence you can not use a standard euclidean metric, and you need to define a custom metric, like we did in the \verb|code/metrics/hybrid_data/| exercise during the course. Compute the mean dissimilarity and the standard deviation of the dissimilarity distribution that you obtain, and save the dissimilarity matrix to a file (e.g. a npy file).

    Importantly, you must define and explain which features are more important with this metric, since you have to balance the contribution of all the features. Your metric should be meaningful in the sense that not all feature values should induce the same contribution to the dissimilarity : the music style "technical death metal" is closer to "metal" than it is to "classical".
\end{Problem}

\begin{Solution}

The columns in the dataset are age, height, job, city, and favorite music style.

The age and height features are numerical and their dissimilarity can be computed using the euclidean distance.

The job, city, and favorite music style features are not numerical so we must define a custom metric for each of them.

\subsection*{Job}

See \verb|job.py| for the code.

Here are the jobs we can find in the dataset: designer, fireman, teacher, doctor, painter, developper, and engineer.

We will assign an art, science, and altruism value from 0 to 10 for each job.

The values given are VERY subjective, we don't mean any offense.

\hfill

\begin{center}
    \begin{tabular}{||c c c c||}
        \hline
        & art & science & altruism \\
        \hline\hline
        designer & 8 & 3 & 4 \\
        \hline
        fireman & 0 & 7 & 10 \\
        \hline
        teacher & 4 & 5 & 6 \\
        \hline
        doctor & 2 & 9 & 8 \\
        \hline
        painter & 10 & 2 & 3 \\
        \hline
        developper & 3 & 6 & 1 \\
        \hline
        engineer & 4 & 8 & 2 \\
        \hline
    \end{tabular}
\end{center}

Using the euclidean distance we get the following dissimilarity matrix:

\begin{center}
    \begin{tabular}{||c c c c c c c c||}
        \hline
        & designer & fireman & teacher & doctor & painter & developper & engineer \\
        \hline\hline
        designer & 0.000000 & 10.770330 & 4.898979 & 9.380832 & 2.449490 & 6.557439 & 6.708204 \\
        \hline
        fireman & 10.770330 & 0.000000 & 6.000000 & 3.464102 & 13.190906 & 9.539392 & 9.000000 \\
        \hline
        teacher & 4.898979 & 6.000000 & 0.000000 & 4.898979 & 7.348469 & 5.196152 & 5.000000 \\
        \hline
        doctor & 9.380832 & 3.464102 & 4.898979 & 0.000000 & 11.747340 & 7.681146 & 6.403124 \\
        \hline
        painter & 2.449490 & 13.190906 & 7.348469 & 11.747340 & 0.000000 & 8.306624 & 8.544004 \\
        \hline
        developper & 6.557439 & 9.539392 & 5.196152 & 7.681146 & 8.306624 & 0.000000 & 2.449490 \\
        \hline
        engineer & 6.708204 & 9.000000 & 5.000000 & 6.403124 & 8.544004 & 2.449490 & 0.000000 \\
        \hline
    \end{tabular}
\end{center}

We can see that jobs like painter and fireman have a high dissimilarity (13.2) while painter and designer have a low dissimilarity (2.4).

\subsection*{City}

See \verb|city.py| for the code.

Here are the cities we can find in the dataset: paris, marseille, toulouse, madrid, and lille.

We will evaluate cities using 4 metrics: distance (using coordinates), population, country, and if it is a capital.

Here is the data we used:

\begin{center}
    \begin{tabular}{||c c c c c||}
        \hline
        & coordinates & population & country & capital \\
        \hline\hline
        paris & (48.8566, 2.3522) & 2161000 & France & True \\
        \hline
        marseille & (43.2965, 5.3698) & 861635 & France & False \\
        \hline
        toulouse & (43.6047, 1.4442) & 471941 & France & False \\
        \hline
        madrid & (40.4168, 3.7038) & 3223000 & Spain & True \\
        \hline
        lille & (50.6292, 3.0573) & 232741 & France & False \\
        \hline
    \end{tabular}
\end{center}

To measure distance we use the library geopy we then use log10 so that it doesn't impact the dissimilarity too much.

We compare the population using euclidean distance. We then use log10 so that it doesn't impact the dissimilarity too much.

Not being from the same country adds a dissimilarity of 10.

One being a capital and not the other adds a dissimilarity of 5.

We then compute the square root of the sum of the squares to get the dissimilarity.

Using this method we get the following dissimilarity matrix:

\begin{center}
    \begin{tabular}{||c c c c c c||}
        \hline
        & paris & marseille & toulouse & madrid & lille \\
        \hline\hline
        paris & 0.000000 & 7.897956 & 7.986461 & 11.675366 & 8.031395 \\
        \hline
        marseille & 7.897956 & 0.000000 & 5.590724 & 12.869235 & 5.798577 \\
        \hline
        toulouse & 7.986461 & 5.590724 & 0.000000 & 12.902215 & 5.378761 \\
        \hline
        madrid & 11.675366 & 12.869235 & 12.902215 & 0.000000 & 12.920325 \\
        \hline
        lille & 8.031395 & 5.798577 & 5.378761 & 12.920325 & 0.000000 \\
        \hline
    \end{tabular}
\end{center}

\subsection*{Favorite music style}

See \verb|music.py| for the code.

Here are the cities we can find in the dataset: trap, hiphop, metal, rock, rap, classical, other, jazz, and technical death metal.

It is hard to find metrics for music styles, we decided to give each pair of music styles a dissimilarity based on personal knowledge.

Some important assumptions that influenced our choices:

\begin{itemize}
    \item trap, hiphop, and rap are related
    \item metal, rock, and technical death metal are related
    \item other is very vague and large, so we gave it a dissimilarity of 10 for all music styles
\end{itemize}

We get this dissimilarity matrix:

\begin{center}
    \begin{tabular}{||c c c c c c c c c c||}
        \hline
        & trap & hiphop & metal & rock & rap & classical & other & jazz & technical death metal \\
        \hline\hline
        trap & 0 & 3 & 20 & 20 & 5 & 20 & 10 & 15 & 20 \\
        \hline
        hiphop & 3 & 0 & 18 & 17 & 5 & 15 & 10 & 12 & 20 \\
        \hline
        metal & 20 & 18 & 0 & 5 & 10 & 14 & 10 & 20 & 5 \\
        \hline
        rock & 20 & 17 & 5 & 0 & 10 & 12 & 10 & 17 & 13 \\
        \hline
        rap & 5 & 5 & 10 & 10 & 0 & 15 & 10 & 15 & 20 \\
        \hline
        classical & 20 & 15 & 14 & 12 & 15 & 0 & 10 & 8 & 20 \\
        \hline
        other & 10 & 10 & 10 & 10 & 10 & 10 & 0 & 10 & 10 \\
        \hline
        jazz & 15 & 12 & 20 & 17 & 15 & 8 & 10 & 0 & 20 \\
        \hline
        technical death metal & 20 & 20 & 5 & 13 & 20 & 20 & 10 & 20 & 0 \\
        \hline
    \end{tabular}
\end{center}

\section*{Overall dissimilarity}

\subsection*{Adjusting means}

See \verb|exercise_2.py| for the code.

If we look at the means and standard deviation for each column we get

\begin{center}
    \begin{tabular}{||c c c||}
        \hline
        & mean & std \\
        \hline\hline
        age & 6.456159 & 4.892174 \\
        \hline
        height & 6.000623 & 4.679549 \\
        \hline
        job & 6.259779 & 3.747122 \\
        \hline
        city & 6.950629 & 5.205568 \\
        \hline
        favorite music style & 11.796500 & 6.390359 \\
        \hline
    \end{tabular}
\end{center}

The mean isn't the same, which means that because of the way we compute dissimilarity some columns have inherently more value, that's an issue. It also makes standard deviations impossible to compare.

We will try to have all means equal 10 (10 is arbitrary, it makes things relatively readable).

After adjustment we get the following:

\begin{center}
    \begin{tabular}{||c c c c||}
        \hline
        & mean & std & adjusted std \\
        \hline\hline
        age & 6.456159 & 4.892174 & 7.577531 \\
        \hline
        height & 6.000623 & 4.679549 & 7.798439 \\
        \hline
        job & 6.259779 & 3.747122 & 5.986029 \\
        \hline
        city & 6.950629 & 5.205568 & 7.489348 \\
        \hline
        favorite music style & 11.796500 & 6.390359 & 5.417165 \\
        \hline
    \end{tabular}
\end{center}

\subsection*{Deciding feature importance}

\textbf{Age}: The age of a person changes a lot of a person, beliefs, physical ability, experience, etc...

$\Rightarrow 3$

\textbf{Height}: Beside appearance and physical ability (in some contexts) this doesn't change much

$\Rightarrow 1$

\textbf{Job}: job is closely related to knowledge, ability, wealth, status, and more

$\Rightarrow 3$

\textbf{City}: Geographical location is related to culture, opportunities, language, and more

$\Rightarrow 2.5$

\textbf{Favorite music style}: As explained before, this dissimilarity is very hard to measure and music styles have a lot of intersections

$\Rightarrow 0.5$

\subsection*{Result matrix}

See \verb|dissimilarity_matrix.npy| for the final dissimilarity matrix.

mean: 58.386

standard deviation: 19.976

\subsubsection*{Side note: most similar and dissimilar items}

Most similar items:
\begin{center}
    \begin{tabular}{||c c c c c c||}
        \hline
        & age & height & job & city & favorite music style \\
        \hline\hline
        102 & 27.086348 & 180.242244 & teacher & madrid & jazz \\
        \hline
        163 & 26.968458 & 179.665081 & teacher & madrid & jazz \\
        \hline
    \end{tabular}
\end{center}

Most dissimilar items:
\begin{center}
    \begin{tabular}{||c c c c c c||}
        \hline
        & age & height & job & city & favorite music style \\
        \hline\hline
        40 & 10.851506 & 169.432515 & fireman & lille & jazz \\
        \hline
        85 & 46.610179 & 181.358551 & fireman & toulouse & trap \\
        \hline
    \end{tabular}
\end{center}

\end{Solution}

%%%%%%%%%%%%%%%%%%%%%%%%%%%%%%%%%%%%%%%%%%%%%%%%%%%%%%%%%%%%%%%%%%
\end{document}
%%%%%%%%%%%%%%%%%%%%%%%%%%%%%%%%%%%%%%%%%%%%%%%%%%%%%%%%%%%%%%%%%%
